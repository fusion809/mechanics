\documentclass[12pt,a4paper,portrait]{article}
%\setcounter{secnumdepth}{0}
\usepackage{gensymb}
\usepackage{pdflscape}
\usepackage{amsmath}
\usepackage{amssymb}
\usepackage{enumitem}
\usepackage{graphicx}
\usepackage{subcaption}
\usepackage{multirow}
\usepackage{sansmath}
\usepackage{pst-eucl}
\usepackage{multicol}
\usepackage{csquotes}
% Coding
\usepackage{listings}
\setlength{\parindent}{0pt}
\usepackage[obeyspaces]{url}
% Better inline directory listings
\usepackage{xcolor}
\definecolor{light-gray}{gray}{0.95}
\newcommand{\code}[1]{\colorbox{light-gray}{\texttt{#1}}}
\usepackage{adjustbox}
\usepackage[UKenglish]{isodate}
\usepackage[UKenglish]{babel}
\usepackage{float}
\usepackage[T1]{fontenc}
\usepackage{setspace}
\usepackage{sectsty}
\usepackage{longtable}
\newenvironment{tightcenter}{%
	\setlength\topsep{0pt}
	\setlength\parskip{0pt}
	\begin{center}
	}{%
	\end{center}
}
\captionsetup{width=\textwidth}
\usepackage{mbenotes} % to print table notes!
\usepackage{alphalph} % For extended counters!
% usage: \tabnotemark[3]\cmsp\tabnotemark[4]
\usepackage[colorlinks=true,linkcolor=blue,urlcolor=black,bookmarksopen=true]{hyperref}
\sectionfont{%			            % Change font of \section command
	\usefont{OT1}{phv}{b}{n}%		% bch-b-n: CharterBT-Bold font
	\sectionrule{0pt}{0pt}{-5pt}{3pt}}
\subsectionfont{
	\usefont{OT1}{phv}{b}{n}}
\newcommand{\MyName}[1]{ % Name
	\usefont{OT1}{phv}{b}{n} \begin{center}of {\LARGE  #1}\end{center}
	\par \normalsize \normalfont}
\makeatletter
\newcommand\FirstWord[1]{\@firstword#1 \@nil}%
\newcommand\@firstword{}%
\newcommand\@removecomma{}%
\def\@firstword#1 #2\@nil{\@removecomma#1,\@nil}%
\def\@removecomma#1,#2\@nil{#1}
\makeatother

\newcommand{\MyTitle}[1]{ % Name
	\Huge \usefont{OT1}{phv}{b}{n} \begin{center}#1\end{center}
	\par \normalsize \normalfont}
\newcommand{\NewPart}[1]{\section*{\uppercase{#1}}}
\newcommand{\NewSubPart}[1]{\subsection*{\hspace{0.2cm}#1}}
\renewcommand{\baselinestretch}{1.05}
\usepackage[margin=0.2cm]{geometry}
\date{}
\title{Elastic pendulum}
\author{Brenton Horne}

\begin{document}
\maketitle

\tableofcontents

\section{Single elastic pendulum with friction}
Say we have a mass $m$ attached to a spring of rest length $l_0$. Suppose we call the displacement from rest $z$. That way the length of the spring is $l(t) = l_0 + z$. If we measure $\theta$ clockwise from the positive x-axis, then our Cartesian coordinates are defined as:

\begin{align*}
	x &= (l_0+z)\cos{\theta} &\implies \dot{x} &= \dot{z}\cos{\theta} - (l_0+z)\dot{\theta}\sin{\theta}\\
	y &= (l_0+z)\sin{\theta} &\implies \dot{y} &= \dot{z}\sin{\theta} + (l_0+z)\dot{\theta}\cos{\theta}.\\
\end{align*}

Therefore:
\begin{align*}
	v^2 &= \dot{x}^2+\dot{y}^2 \\
	&= \left[\dot{z}\cos{\theta} - (l_0+z)\dot{\theta}\sin{\theta}\right]^2 + \left[\dot{z}\sin{\theta} + (l_0+z)\dot{\theta}\cos{\theta}\right]^2 \\
	&= \dot{z}^2 \cos^2{\theta} + (l_0+z)^2\dot{\theta}^2\sin^2{\theta} - 2\dot{z}\dot{\theta}(l_0+z)\cos{\theta}\sin{\theta} + \dot{z}^2\sin^2{\theta} + (l_0+z)^2\dot{\theta}^2\cos^2{\theta} + 2\dot{z}\dot{\theta}(l_0+z)\sin{\theta}\cos{\theta} \\
	&= \dot{z}^2 + (l_0+z)^2\dot{\theta}^2.
\end{align*}

Hence the kinetic energy is:
\begin{align*}
	T &= \dfrac{mv^2}{2} \\
	&= \dfrac{m}{2} \left[\dot{z}^2 + (l_0+z)^2\dot{\theta}^2\right].
\end{align*}

As for the potential energy, it will have two components: a spring component,
\begin{align*}
	V_S &= \dfrac{kz^2}{2};
\end{align*}

and a graphical component,
\begin{align*}
	V_G &= mgy \\
	&= mg(l_0+z)\sin{\theta}.
\end{align*}

Therefore the total potential energy is:
\begin{align*}
	V &= V_S + V_G \\
	&= \dfrac{kz^2}{2} + mg(l_0+z)\sin{\theta}.
\end{align*}

Hence the Lagrangian is:
\begin{align*}
	\mathcal{L} &= T - V \\
	&= \dfrac{m}{2} \left[\dot{z}^2 + \dot{\theta}^2(l_0+z)^2\right] - \dfrac{kz^2}{2} - mg(l_0+z)\sin{\theta}.
\end{align*}

To find the equations of motion, we use the Euler-Lagrange equations with a dissipative force:
\begin{align}
	\dfrac{d}{dt}\left(\dfrac{\partial \mathcal{L}}{\partial \dot{q}_i}\right) - \dfrac{\partial \mathcal{L}}{\partial q_i} &= Q_i. \label{ELD}
\end{align}

Where $Q_i$ is the generalized dissipative force calculated using:
\begin{align*}
	Q_i &= \vec{F}_{D} \cdot \dfrac{\partial \vec{r}}{\partial q_i}.
\end{align*}

Where $\vec{F}_D$ is the dissipative force. In this case, our generalized coordinates $q_i$ would consist of $\theta$ and $z$. 

What dissipative forces should be include? Let us include linear and quadratic in velocity terms to account for air resistance and other forms of friction. 

\begin{align*}
	\vec{F}_D &= -b\vec{v} - c|\vec{v}|\vec{v}.
\end{align*}

We have already calculate the components of the velocity vector---they are $\dot{x}$ and $\dot{y}$, respectively. 

\begin{align*}
	\vec{v} &= \begin{bmatrix}
		\dot{z}\cos{\theta} - (l_0+z)\dot{\theta}\sin{\theta} \\
		\dot{z}\sin{\theta} + (l_0+z)\dot{\theta}\cos{\theta}
	\end{bmatrix}\\
	|\vec{v}| &= \sqrt{v^2}\\
	&= \sqrt{\dot{z}^2+(l_0+z)^2\dot{\theta}^2}.
\end{align*}

Next we will calculate $\dfrac{\partial \vec{r}}{\partial q_i}$:

\begin{align*}
	\dfrac{\partial \vec{r}}{\partial \theta} &= (l_0+z)\begin{bmatrix}
		-\sin{\theta} \\
		\cos{\theta}
	\end{bmatrix} \\
	\dfrac{\partial \vec{r}}{\partial z} &= \begin{bmatrix}
		\cos{\theta} \\
		\sin{\theta}
	\end{bmatrix}.
\end{align*}

To calculate, $Q_i$ we must find $\vec{v} \cdot \dfrac{\partial \vec{r}}{\partial q_i}$:

\begin{align*}
	\vec{v} \cdot \dfrac{\partial \vec{r}}{\partial \theta} &= \begin{bmatrix}
		\dot{z}\cos{\theta} - (l_0+z)\dot{\theta}\sin{\theta} \\
		\dot{z}\sin{\theta} + (l_0+z)\dot{\theta}\cos{\theta}
	\end{bmatrix} \cdot (l_0+z)\begin{bmatrix}
		-\sin{\theta} \\
		\cos{\theta}
	\end{bmatrix} \\
	&= -\dot{z}(l_0+z)\cos{\theta} \sin{\theta} + (l_0+z)^2 \dot{\theta}\sin^2{\theta} + \dot{z}(l_0+z)\sin{\theta} \cos{\theta} +(l_0+z)^2\dot{\theta}\cos^2{\theta} \\
	&= (l_0+z)^2 \dot{\theta} \\
	\vec{v} \cdot \dfrac{\partial \vec{r}}{\partial z} &= \begin{bmatrix}
		\dot{z}\cos{\theta} - (l_0+z)\dot{\theta}\sin{\theta} \\
		\dot{z}\sin{\theta} + (l_0+z)\dot{\theta}\cos{\theta}
	\end{bmatrix} \cdot \begin{bmatrix}
		\cos{\theta} \\
		\sin{\theta}
	\end{bmatrix} \\
	&= \dot{z}\cos^2{\theta} - (l_0+z)\dot{\theta} \sin{\theta}\cos{\theta} + \dot{z}\sin^2{\theta} + (l_0+z)\dot{\theta}\cos{\theta}\sin{\theta} \\
	&= \dot{z}. 
\end{align*}

Hence $Q_i$ is:

\begin{align*}
	Q_{\theta} &= -(l_0+z)^2 \dot{\theta} \left(b+c\sqrt{\dot{z}^2+(l_0+z)^2\dot{\theta}^2}\right) \\
	Q_z &= -\dot{z}\left(b+c\sqrt{\dot{z}^2+(l_0+z)^2\dot{\theta}^2}\right).
\end{align*}

As for the LHS of Equation \eqref{ELD}, let us work on it for $\theta$ one term at a time:
\begin{align*}
	p_{\theta} &= \dfrac{\partial \mathcal{L}}{\partial \dot{\theta}} \\
	&= m\dot{\theta}(l_0+z)^2 \\
	\dot{p_{\theta}} &= m\ddot{\theta} (l_0+z)^2 + 2m\dot{\theta}\dot{z}(l_0+z) \\
	F_{\theta} &= \dfrac{\partial \mathcal{L}}{\partial \theta} \\
	&= -mg(l_0+z)\cos{\theta}.
\end{align*}

Substituting into Equation \eqref{ELD} yields:
\begin{align*}
	m\ddot{\theta} (l_0+z)^2 + 2m\dot{\theta}\dot{z}(l_0+z) - (-mg(l_0+z)\cos{\theta}) &= -(l_0+z)^2 \dot{\theta} \left(b+c\sqrt{\dot{z}^2+(l_0+z)^2\dot{\theta}^2}\right) \\
	m\ddot{\theta} (l_0+z)^2 + 2m\dot{\theta}\dot{z}(l_0+z) +mg(l_0+z)\cos{\theta} &= -(l_0+z)^2 \dot{\theta} \left(b+c\sqrt{\dot{z}^2+(l_0+z)^2\dot{\theta}^2}\right) \\
	\ddot{\theta} &= -\dfrac{2\dot{\theta}\dot{z}}{l_0+z} - \dfrac{g\cos{\theta}}{l_0+z} -\dfrac{\dot{\theta}}{m} \left(b+c\sqrt{\dot{z}^2+(l_0+z)^2\dot{\theta}^2}\right). \\
\end{align*}

Let $k'=\dfrac{k}{m}$, $b'=\dfrac{b}{m}$ and $c'=\dfrac{c}{m}$.

\begin{align*}
	\ddot{\theta} &= -\dfrac{2\dot{\theta}\dot{z}}{l_0+z} - \dfrac{g\cos{\theta}}{l_0+z} -\dot{\theta} \left(b'+c'\sqrt{\dot{z}^2+(l_0+z)^2\dot{\theta}^2}\right).
\end{align*}
As for $z$:

\begin{align*}
	p_z &= \dfrac{\partial \mathcal{L}}{\partial \dot{z}} \\
	&= m\dot{z} \\
	\dot{p_z} &= m\ddot{z} \\
	F_z &= \dfrac{\partial \mathcal{L}}{\partial z} \\
	&= m\dot{\theta}^2(l_0+z) - kz - mg\sin{\theta}.
\end{align*}

Substituting into Equation \eqref{ELD} yields:

\begin{align*}
	m\ddot{z} - m\dot{\theta}^2(l_0+z) + kz + mg\sin{\theta} &= -\dot{z}\left(b+c\sqrt{\dot{z}^2+(l_0+z)^2\dot{\theta}^2}\right) \\
	\ddot{z} &= \dot{\theta}^2(l_0+z) - \dfrac{kz}{m} - g\sin{\theta} -\dfrac{\dot{z}}{m}\left(b+c\sqrt{\dot{z}^2+(l_0+z)^2\dot{\theta}^2}\right).
\end{align*}

Let $k'=\dfrac{k}{m}$, $b'=\dfrac{b}{m}$ and $c'=\dfrac{c}{m}$.

\begin{align*}
	\ddot{z} &= \dot{\theta}^2(l_0+z) - k'z - g\sin{\theta} -\dot{z}\left(b'+c'\sqrt{\dot{z}^2+(l_0+z)^2\dot{\theta}^2}\right).
\end{align*}
	
\end{document}